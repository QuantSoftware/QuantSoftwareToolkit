%
% API Documentation for QSTK
% Module QSTK.qstklearn.fastknn
%
% Generated by epydoc 3.0.1
% [Mon Mar  5 00:49:20 2012]
%

%%%%%%%%%%%%%%%%%%%%%%%%%%%%%%%%%%%%%%%%%%%%%%%%%%%%%%%%%%%%%%%%%%%%%%%%%%%
%%                          Module Description                           %%
%%%%%%%%%%%%%%%%%%%%%%%%%%%%%%%%%%%%%%%%%%%%%%%%%%%%%%%%%%%%%%%%%%%%%%%%%%%

    \index{QSTK \textit{(package)}!QSTK.qstklearn \textit{(package)}!QSTK.qstklearn.fastknn \textit{(module)}|(}
\section{Module QSTK.qstklearn.fastknn}

    \label{QSTK:qstklearn:fastknn}
This package is an implementation of a novel improvement to KNN which 
speeds up query times


%%%%%%%%%%%%%%%%%%%%%%%%%%%%%%%%%%%%%%%%%%%%%%%%%%%%%%%%%%%%%%%%%%%%%%%%%%%
%%                               Functions                               %%
%%%%%%%%%%%%%%%%%%%%%%%%%%%%%%%%%%%%%%%%%%%%%%%%%%%%%%%%%%%%%%%%%%%%%%%%%%%

  \subsection{Functions}

    \label{QSTK:qstklearn:fastknn:adistfun}
    \index{QSTK \textit{(package)}!QSTK.qstklearn \textit{(package)}!QSTK.qstklearn.fastknn \textit{(module)}!QSTK.qstklearn.fastknn.adistfun \textit{(function)}}

    \vspace{0.5ex}

\hspace{.8\funcindent}\begin{boxedminipage}{\funcwidth}

    \raggedright \textbf{adistfun}(\textit{u}, \textit{v})

\setlength{\parskip}{2ex}
\setlength{\parskip}{1ex}
    \end{boxedminipage}

    \label{QSTK:qstklearn:fastknn:dataifywine}
    \index{QSTK \textit{(package)}!QSTK.qstklearn \textit{(package)}!QSTK.qstklearn.fastknn \textit{(module)}!QSTK.qstklearn.fastknn.dataifywine \textit{(function)}}

    \vspace{0.5ex}

\hspace{.8\funcindent}\begin{boxedminipage}{\funcwidth}

    \raggedright \textbf{dataifywine}(\textit{fname})

\setlength{\parskip}{2ex}
\setlength{\parskip}{1ex}
    \end{boxedminipage}

    \label{QSTK:qstklearn:fastknn:testwine}
    \index{QSTK \textit{(package)}!QSTK.qstklearn \textit{(package)}!QSTK.qstklearn.fastknn \textit{(module)}!QSTK.qstklearn.fastknn.testwine \textit{(function)}}

    \vspace{0.5ex}

\hspace{.8\funcindent}\begin{boxedminipage}{\funcwidth}

    \raggedright \textbf{testwine}()

\setlength{\parskip}{2ex}
\setlength{\parskip}{1ex}
    \end{boxedminipage}

    \label{QSTK:qstklearn:fastknn:testspiral}
    \index{QSTK \textit{(package)}!QSTK.qstklearn \textit{(package)}!QSTK.qstklearn.fastknn \textit{(module)}!QSTK.qstklearn.fastknn.testspiral \textit{(function)}}

    \vspace{0.5ex}

\hspace{.8\funcindent}\begin{boxedminipage}{\funcwidth}

    \raggedright \textbf{testspiral}()

\setlength{\parskip}{2ex}
\setlength{\parskip}{1ex}
    \end{boxedminipage}

    \label{QSTK:qstklearn:fastknn:getflatcsv}
    \index{QSTK \textit{(package)}!QSTK.qstklearn \textit{(package)}!QSTK.qstklearn.fastknn \textit{(module)}!QSTK.qstklearn.fastknn.getflatcsv \textit{(function)}}

    \vspace{0.5ex}

\hspace{.8\funcindent}\begin{boxedminipage}{\funcwidth}

    \raggedright \textbf{getflatcsv}(\textit{fname})

\setlength{\parskip}{2ex}
\setlength{\parskip}{1ex}
    \end{boxedminipage}

    \label{QSTK:qstklearn:fastknn:testgendata}
    \index{QSTK \textit{(package)}!QSTK.qstklearn \textit{(package)}!QSTK.qstklearn.fastknn \textit{(module)}!QSTK.qstklearn.fastknn.testgendata \textit{(function)}}

    \vspace{0.5ex}

\hspace{.8\funcindent}\begin{boxedminipage}{\funcwidth}

    \raggedright \textbf{testgendata}()

\setlength{\parskip}{2ex}
\setlength{\parskip}{1ex}
    \end{boxedminipage}

    \label{QSTK:qstklearn:fastknn:test}
    \index{QSTK \textit{(package)}!QSTK.qstklearn \textit{(package)}!QSTK.qstklearn.fastknn \textit{(module)}!QSTK.qstklearn.fastknn.test \textit{(function)}}

    \vspace{0.5ex}

\hspace{.8\funcindent}\begin{boxedminipage}{\funcwidth}

    \raggedright \textbf{test}()

\setlength{\parskip}{2ex}
\setlength{\parskip}{1ex}
    \end{boxedminipage}


%%%%%%%%%%%%%%%%%%%%%%%%%%%%%%%%%%%%%%%%%%%%%%%%%%%%%%%%%%%%%%%%%%%%%%%%%%%
%%                               Variables                               %%
%%%%%%%%%%%%%%%%%%%%%%%%%%%%%%%%%%%%%%%%%%%%%%%%%%%%%%%%%%%%%%%%%%%%%%%%%%%

  \subsection{Variables}

    \vspace{-1cm}
\hspace{\varindent}\begin{longtable}{|p{\varnamewidth}|p{\vardescrwidth}|l}
\cline{1-2}
\cline{1-2} \centering \textbf{Name} & \centering \textbf{Description}& \\
\cline{1-2}
\endhead\cline{1-2}\multicolumn{3}{r}{\small\textit{continued on next page}}\\\endfoot\cline{1-2}
\endlastfoot\raggedright \_\-\_\-p\-a\-c\-k\-a\-g\-e\-\_\-\_\- & \raggedright \textbf{Value:} 
{\tt \texttt{'}\texttt{QSTK.qstklearn}\texttt{'}}&\\
\cline{1-2}
\end{longtable}


%%%%%%%%%%%%%%%%%%%%%%%%%%%%%%%%%%%%%%%%%%%%%%%%%%%%%%%%%%%%%%%%%%%%%%%%%%%
%%                           Class Description                           %%
%%%%%%%%%%%%%%%%%%%%%%%%%%%%%%%%%%%%%%%%%%%%%%%%%%%%%%%%%%%%%%%%%%%%%%%%%%%

    \index{QSTK \textit{(package)}!QSTK.qstklearn \textit{(package)}!QSTK.qstklearn.fastknn \textit{(module)}!QSTK.qstklearn.fastknn.FastKNN \textit{(class)}|(}
\subsection{Class FastKNN}

    \label{QSTK:qstklearn:fastknn:FastKNN}
A class which implements the KNN learning algorithm with sped up query 
times.

This class follows the conventions of other classes in the qstklearn 
module, with a constructor that initializes basic parameters and 
bookkeeping variables, an 'addEvidence' method for adding labeled training 
data individually or as a batch, and a 'query' method that returns an 
estimated class for an unlabled point.  Training and testing data are in 
the form of numpy arrays, and classes are discrete.

In order to speed up query times, this class keeps a number of lists which 
sort the training data by distance to 'anchor' points.  The lists aren't 
sorted until the first call to the 'query' method, after which, the lists 
are kept in sorted order. Initial sort is done using pythons 'sort' 
(samplesort), and sorted insertions with 'insort' from the bisect module.


%%%%%%%%%%%%%%%%%%%%%%%%%%%%%%%%%%%%%%%%%%%%%%%%%%%%%%%%%%%%%%%%%%%%%%%%%%%
%%                                Methods                                %%
%%%%%%%%%%%%%%%%%%%%%%%%%%%%%%%%%%%%%%%%%%%%%%%%%%%%%%%%%%%%%%%%%%%%%%%%%%%

  \subsubsection{Methods}

    \label{QSTK:qstklearn:fastknn:FastKNN:__init__}
    \index{QSTK \textit{(package)}!QSTK.qstklearn \textit{(package)}!QSTK.qstklearn.fastknn \textit{(module)}!QSTK.qstklearn.fastknn.FastKNN \textit{(class)}!QSTK.qstklearn.fastknn.FastKNN.\_\_init\_\_ \textit{(method)}}

    \vspace{0.5ex}

\hspace{.8\funcindent}\begin{boxedminipage}{\funcwidth}

    \raggedright \textbf{\_\_init\_\_}(\textit{self}, \textit{num\_anchors}, \textit{k})

    \vspace{-1.5ex}

    \rule{\textwidth}{0.5\fboxrule}
\setlength{\parskip}{2ex}
    Creates a new FastKNN object that will use the given number of anchors.

\setlength{\parskip}{1ex}
    \end{boxedminipage}

    \label{QSTK:qstklearn:fastknn:FastKNN:resetAnchors}
    \index{QSTK \textit{(package)}!QSTK.qstklearn \textit{(package)}!QSTK.qstklearn.fastknn \textit{(module)}!QSTK.qstklearn.fastknn.FastKNN \textit{(class)}!QSTK.qstklearn.fastknn.FastKNN.resetAnchors \textit{(method)}}

    \vspace{0.5ex}

\hspace{.8\funcindent}\begin{boxedminipage}{\funcwidth}

    \raggedright \textbf{resetAnchors}(\textit{self}, \textit{selection\_type}={\tt \texttt{'}\texttt{random}\texttt{'}})

    \vspace{-1.5ex}

    \rule{\textwidth}{0.5\fboxrule}
\setlength{\parskip}{2ex}
    Picks a new set of anchors.  The anchor lists will be re-sorted upon 
    the next call to 'query'.

    selection\_type - the method to use when selecting new anchor points. 
    'random' performs a random permutation of the training points and picks
    the first 'num\_anchors' as new anchors.

\setlength{\parskip}{1ex}
    \end{boxedminipage}

    \label{QSTK:qstklearn:fastknn:FastKNN:addEvidence}
    \index{QSTK \textit{(package)}!QSTK.qstklearn \textit{(package)}!QSTK.qstklearn.fastknn \textit{(module)}!QSTK.qstklearn.fastknn.FastKNN \textit{(class)}!QSTK.qstklearn.fastknn.FastKNN.addEvidence \textit{(method)}}

    \vspace{0.5ex}

\hspace{.8\funcindent}\begin{boxedminipage}{\funcwidth}

    \raggedright \textbf{addEvidence}(\textit{self}, \textit{data}, \textit{label})

    \vspace{-1.5ex}

    \rule{\textwidth}{0.5\fboxrule}
\setlength{\parskip}{2ex}
    Adds to the set of training data. If the anchor lists were sorted 
    before the call to this method, the new data will be inserted into the 
    anchor lists using 'bisect.insort'

    data - a numpy array, either a single point (1D) or a set of points 
    (2D)

    label - the label for data. A single value, or a list of values in the 
    same order as the points in data.

\setlength{\parskip}{1ex}
    \end{boxedminipage}

    \label{QSTK:qstklearn:fastknn:FastKNN:query}
    \index{QSTK \textit{(package)}!QSTK.qstklearn \textit{(package)}!QSTK.qstklearn.fastknn \textit{(module)}!QSTK.qstklearn.fastknn.FastKNN \textit{(class)}!QSTK.qstklearn.fastknn.FastKNN.query \textit{(method)}}

    \vspace{0.5ex}

\hspace{.8\funcindent}\begin{boxedminipage}{\funcwidth}

    \raggedright \textbf{query}(\textit{self}, \textit{point}, \textit{k}={\tt None}, \textit{method}={\tt \texttt{'}\texttt{mode}\texttt{'}}, \textit{slow}={\tt False}, \textit{dumdumcheck}={\tt False})

    \vspace{-1.5ex}

    \rule{\textwidth}{0.5\fboxrule}
\setlength{\parskip}{2ex}
    Returns class value for an unlabled point by examining its k nearest 
    neighbors. 'method' determines how the class of the unlabled point is 
    determined.

\setlength{\parskip}{1ex}
    \end{boxedminipage}

    \index{QSTK \textit{(package)}!QSTK.qstklearn \textit{(package)}!QSTK.qstklearn.fastknn \textit{(module)}!QSTK.qstklearn.fastknn.FastKNN \textit{(class)}|)}
    \index{QSTK \textit{(package)}!QSTK.qstklearn \textit{(package)}!QSTK.qstklearn.fastknn \textit{(module)}|)}
